\documentclass[a4paper,11pt,openany,notitlepage]{book}
\usepackage{amsmath}
\usepackage{amsfonts}
\usepackage{amssymb}
\usepackage[UTF8]{ctex}
\usepackage{graphicx}
\usepackage{color}

\usepackage{titlesec}
\titleformat{\section}{\bfseries\Large}{$\S$\,\arabic{section}}{1em}{}
\titleformat{\subsection}{\bfseries\large}{\Roman{subsection}}{1em}{}
\titlespacing*{\subsection}{2em}{2pt}{2pt}
\title{\vspace{-1.5cm} \textbf{\huge{数值分析第一章上机}}\vspace{-1em}}
\author{By 211870125 陈睿硕}
\date{\vspace{-0.5cm}2023.2.28}

\usepackage{geometry}
\geometry{left=2cm,right=2cm,top=2cm,bottom=2cm}

\usepackage{fancyhdr}
\pagestyle{fancy}
\fancyhf{}
\fancyhead[L]{Week 1}
\fancyhead[R]{\thepage}

\usepackage{listings}  % 引入 listings 包
\lstset{                % 定义代码块的样式
    basicstyle=\normalsize\ttfamily, % 设定代码字体大小、样式
    language=C++,    % 设定编程语言
    showspaces=false,   % 不显示空格
    showstringspaces=false, % 不显示字符串中的空格
    showtabs=false,     % 不显示制表符
    frame=single,       % 设定代码块边框样式
    rulecolor=\color{black}, % 设定代码块边框颜色
    tabsize=4,          % 设定制表符长度为 4 个字符
    captionpos=b,       % 设定标题位置为底部
    keywordstyle=\bfseries\color{blue}\ttfamily,
    stringstyle=\color{red}\ttfamily,
    commentstyle=\color{green}\ttfamily,
    morecomment=[l][\color{magenta}]{\#},
    framesep=0.5em,
    frameround=tttt,
    breaklines=true,    % 自动换行
    breakatwhitespace=false, % 只在空格分割处换行
    escapeinside={\%*}{*)}   % 允许使用 LaTeX 命令
}

\begin{document}

\maketitle
\vspace{-1cm}
\thispagestyle{fancy}

\section{问题}
求计算机的规范化浮点数的上溢值($OFL$)、下溢值($UFL$)和计算机的机器精度($\varepsilon_{mach}$)。

\section{算法思路}
问题即求解$-L$(即单精度数最小指数)、$U$(即单精度数最大指数)、$t$(即单精度数精度)。
这是因为利用公式:
\begin{gather}
    OFL = 2^{U} (2 - 2^{-(t-1)}), \label{Eq1.1}\\
    UFL = 2^{-L}, \label{Eq1.2}\\
    \varepsilon_{mach} = 2^{-t}, \label{Eq1.3}
\end{gather}
可以求得问题之解。\\
\indent下面我们将分三步求解上溢值,下溢值和机器精度。
\subsection{求解$t + L$}
我们将$a = 1$不断进行"$\div 2$"操作,直到其变为0,记录下操作的次数$count$。\\
\indent在$a$变为0的前一刻,$a$的数值应该是单精度浮点数的最小正值,即$2^{-L-(t-1)}$。
这是因为非规格数的指数始终默认为$-L$。
从$a = 1$到$a = 2^{-L-(t-1)}$共进行了$L + t - 1$步,故最后当$a$恰变为$0$时,共进行了$t + L$步。
则有:
\begin{equation}
    t + L = count = 150, \label{Eq1.4}  
\end{equation}

\subsection{求解t}
我们令$c = 3,d = 1$,然后不断对两者进行“$\times 2 + 1$"操作,直到式 $c - 1 = 2b$ 不成立,
记录下操作的次数,其即为$t - 1$。\\
\indent这是因为规格数的有效数字不能超过$t$位,一旦超过$t$位便会产生舍入误差。易见每次操作都将$c$、$d$的有效位数增多一位,
当式 $c - 1 = 2b$ 第一次不成立时,$c$的有效数字恰超过$t$位,$d$的有效数字恰为$t$位,故实际上有:
\begin{equation}
    d = (\underbrace{11\cdots\cdots1}_{t\text{个}})_{2} = 2^{t} - 1, \label{Eq1.5}
\end{equation}
此时也正好进行了$t - 1$次操作,如此可求得$t = 24$。

\subsection{求解上溢值}
结合(\ref{Eq1.5})改写(\ref{Eq1.1})为:
\[
    OFL = 2^{U-t+1} (2^{t} - 1) = 2^{U-t+1} d
\]
故类似于\MakeUppercase{\romannumeral 2}中的方法,我们不断将$d$进行“$\times 2$"操作,直至发生舍入。
记录下舍入前一刻的d的值,即为上溢值$OFL$。求得上溢值$OFL \approx 3.40282 \times 10^{38}$。

\subsection{求解下溢值及机器精度}
由$t = 24$及(\ref{Eq1.4})知$L = 150 - t = 126$,利用(\ref{Eq1.2})计算得$UFL \approx 1.17549 \times 10^{-38}$。\\
\indent由(\ref{Eq1.3})计算得$\varepsilon_{mach} \approx 5.96046 \times 10^{-8}$

\section{结果分析}
这个方法是通过找出$t$、$U$、$L$的值利用既知公式计算上溢值、下溢值及机器精度的,故和理论分析的上述三值完全相等。
利用C++的char指针打印出所算出的三值的单精度表示,也和我们理论分析的形式无异,故正确性是肯定的。

\section{附录:程序代码}
\begin{lstlisting}
#include<iostream>
#include<cmath>
using namespace std;
int main()
{
    float a=1,b=1;
    int t_plus_L=0;
    while(a!=0)
    {
        b=a;
        a/=2;
        t_plus_L++;
    }
    float c=3,d=1,t=1;
    while(c-1==2*d)
    {
        d=2*d+1;
        c=2*c+1;
        t++;
    }
    float u=2*d;
    while(u/2==d)
    {
        u*=2;d*=2;
    }
    cout<<"Upflow value:"<<d<<endl<<"Underflow value:"
    <<pow(2,t-t_plus_L)<<endl<<"Machine precision:"<<pow(2,-t)<<endl;
    return 0;
}
\end{lstlisting}
\end{document}